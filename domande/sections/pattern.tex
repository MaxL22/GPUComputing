% !TeX spellcheck = it_IT
\section{Pattern}

\begin{questions}
    \question Pseudocodice di Jones-Plassman per la colorazione parallela
    
    \begin{solution}
        L'algoritmo di Jones-Plassman è un algoritmo approssimato per ambienti distribuiti per risolvere il problema di graph coloring, con una qualità delle soluzioni simile a quella degli algoritmi sequenziali.
        
        Dato un grafo $G = (V, E)$ e un insieme di colori $\mathcal{C}$: 
        \begin{itemize}
            \item Ogni vertice $v \in V$ riceve una priorità casuale
            
            \item Ogni vertice può decidere il proprio colore quando ha priorità maggiore di tutti i vicini non ancora colorati
            
            \item Viene iterato lo step di decisione, fino a completamento
        \end{itemize}
        
        All'interno di ogni round, il passo di decisione e selezione può essere svolto in parallelo. Per un grafo sparso e con distribuzioni casuali dei pesi converge con \textit{alta probabilità} per $O(\log |V|)$.
        
        Pseudocodice (inventato al momento):
        \begin{center}
            \begin{minipage}{.7\textwidth}
                \begin{tcolorbox}[
                    colback=white,
                    sharp corners,
                    boxrule=.3mm,
                    left=20pt,
                    top=0pt,
                    bottom=0pt,
                    colbacktitle=white,
                    coltitle=black
                    ]
                    \LinesNumbered
                    \begin{algorithm}[H]
                        \SetAlgoNoEnd
                        \SetKwSty{texttt}
                        \SetArgSty{relax}
                        $S \leftarrow \emptyset$; \\
                        $R \leftarrow V$;  \\
                        \While{$R \neq \emptyset$}{
                            $\forall v \in R$, $\pi (v) \leftarrow$ rand() \tcp*{parallelo}
                            \If{$\pi (v)> \pi(n)$, $\forall n \in N(v)$ non colorato}{
                                $C \leftarrow \bigcup_{n \in N(v)} c(n)$; \\
                                $c(v) \leftarrow$ colore minimo $\notin C$; \\
                                $S \leftarrow S \cup \{v\}$; \\
                                $R \leftarrow R - \{v\}$; \\
                            }                            
                        }
                    \end{algorithm}
                \end{tcolorbox}
            \end{minipage}
        \end{center}
        
        Dove $\pi: V \rightarrow \N$ restituisce la priorità del nodo e $c: V \rightarrow \mathcal{C}$ restituisce il colore del nodo.
    \end{solution}
    
    \question Illustrare il concetto di parallel reduction.
    
    \begin{solution}
        La parallel reduction è un pattern di programmazione parallela comune che consiste nell'eseguire un'operazione commutativa e associativa su un vettore $a = \{a_0, \dots, a_{n-1}\}$ tale da ottenere il vettore risultato $b = \{a_0, (a_0 \oplus a_1), \dots, (a_0 \oplus \dots \oplus a_{n-1})\}$. In sintesi:
        $$ b[k] = \sum_{i=0}^{k} a[i] \quad \forall k \in 0, \dots, n-1 $$
        
        L'approccio sequenziale è semplice, in parallelo la strategia di base prevede: 
        \begin{itemize}
            \item suddividere il vettore di input
            
            \item assegnare thread per calcolare somme parziali
            
            \item combinare i risultati parziali
        \end{itemize}
        
        I due approcci "naive" (presupponendo che la lunghezza $n$ del vettore in input sia potenza di 2) possibili sono: 
        \begin{itemize}
            \item coppie contigue: a ogni iterazione $i$ un thread somma elementi a distanza $2^i$ (partendo da $i=0$)
            
            \item coppie equispaziate: a ogni iterazione $i$ un thread somma elementi a distanza $n/2^{i+1}$ (partendo da $i=0$)
        \end{itemize}
        
        Un problema di performance che può sorgere con questi approcci è la divergenza crescente a livello di warp se non si determinano in maniera oculata quali thread sono attivi per fare quali somme, l'approccio ingenuo introduce divergenza crescente disattivando metà dei thread attivi a ogni step.
    \end{solution}
    
    \question Implementazione di Horn per prefix sum.
    
    \begin{solution}
        Avendo a disposizione molti processori, si vuole ridurre il tempo delle somme prefisse a $O(\log n)$, anche a costo di qualche operazione in più.
        
        Viene usato un approccio iterativo in $d = \lceil \log_2 n \rceil$ passi, a ogni passo, da $1$ a $d$:
        \begin{itemize}
            \item Calcolo l'offset $\Delta = 2^i$
            
            \item Per ogni elemento $k$ dell'array (in parallelo)
            \begin{itemize}
                \item se $k \geq \Delta$: $x[k] \leftarrow x[k] + x[k - \Delta]$
                
                \item se $k < \Delta$: $x[k] \leftarrow x[k]$
            \end{itemize}
        \end{itemize}
        
        Questo algoritmo effettua $\Theta (\log n)$ step, ma $\Theta (n \log n)$ operazioni, quindi non è work efficient.
    \end{solution}
    
    \question Schema per l'implementazione work efficient della prefix sum.
    
    \begin{solution}
        La prefix sum, dato un operatore binario associativo (in questo caso la somma) e due vettori di dimensione $n$ ($a$ input, $b$ output) vuole fare in modo che:
        $$ b[k] = \sum_{i = 0}^{k} a[i] \quad \forall k \in [0,n-1] $$
        
        La strategia work efficient usa (concettualmente) un albero bilanciato sui dati in input, in cui le foglie sono i valori dell'array di input.
        
        Due fasi: 
        \begin{enumerate}
            \item reduce o up-sweep: per costruire le somme parziali fino alla radice, ogni nodo a livelli intermedi sarà la somma dei nodi figlio
            
            \item down-sweep: per distribuire i prefissi corretti alle foglie; si scorrono i livelli dalla radice verso le foglie, si setta la radice a zero e
            \begin{itemize}
                \item ogni figlio sinistro sarà pari al padre
                
                \item ogni figlio destro sarà la somma del padre e del valore (prima che venga aggiornato) del figlio sinistro
            \end{itemize}
        \end{enumerate}
        
        In totale sono: 
        \begin{itemize}
            \item $2 \log n$ passi, ovvero $\log n$ per fase
            
            \item $2n - 1$ operazioni, una per nodo interno dell'albero per fase
        \end{itemize}
    \end{solution}
    
    \question Spiega il bitonic MergeSort
    
    \begin{solution}
        Una sequenza bitonica è una sequenza $s = \{a_0, \dots, a_{n-1}\}$ su cui vale la proprietà
        $$ \exists i \in [0, n-1] \tc a_0 \leq \dots \leq a_i \geq \dots a_{n-1} $$
        oppure esiste una permutazione ciclica degli indici per il quale la proprietà vale.
        
        Se la sequenza $s = \{a_0, \dots, a_{n-1}\}$ è bitonica, allora
        \begin{align*}
            s_1 & = \{\min(a_0, a_{n/2}), \min(a_1, a_{n/2 + 1}), \dots, \min (a_{n/2 - 1}, a_{n-1})\} \\
            s_2 & = \{\max(a_0, a_{n/2}), \max(a_1, a_{n/2 + 1}), \dots, \max (a_{n/2 - 1}, a_{n-1})\} \\
        \end{align*}
        sono anch'esse sequenze bitoniche e tutti gli elementi di $s_1$ sono minori di tutti gli elementi di $s_2$.
        
        Si può ripetere questo passaggio fino a sequenze di 2 elementi, ordinabili banalmente ($\log n - 1$ passi).
    \end{solution}
    
    \question Spiegare la bitonic merge network.
    
    \begin{solution}
        Una bitonic merging network con una sequenza bitonica lunga $n$ in input produce una sequenza 
        \begin{itemize}
            \item crescente con il comparatore $\oplus BM[n]$
            \item decrescente con il comparatore $\ominus BM[n]$
        \end{itemize}
        
        Per una rete con $n=2$ si tratta di comparatori che semplicemente "scambiano" i valori se sono nell'ordine sbagliato. 
        
        Per una rete con $n$ fili, una $BM[n]$ ha
        \begin{itemize}
            \item $\log_2 n$ colonne
            
            \item $n/2$ comparatori a coppie per colonna
        \end{itemize}
        
        Ogni colonna ha comparatori a distanza metà di quella precedente, a partire da distanza $n/2$, fino a $1$ (dopo $\log_2 n$ colonne). In questo modo, a partire da una sequenza bitonica in entrata, si ha una sequenza ordinata in uscita.
    \end{solution}
    
    \question Spiega l'ordinamento bitonico
    
    \begin{solution}
        A partire da una sequenza qualsiasi, si può creare una sequenza bitonica tramite una rete con $\log_2 n - 1$ colonne in cui ogni $i$-esima colonna usa $\oplus BM [2^i]$ e $\ominus BM [2^i]$ alternati su tutto l'input per creare sequenze bitoniche lunghe $2^{i+1}$ (a partire da $i=0$, un valore singolo soddisfa banalmente le proprietà di una sequenza bitonica).
        
        Alla fine si ha una sequenza bitonica ordinabile con una bitonic merging network $\oplus BM [n]$ (o $\ominus BM [n]$).
        
        Per una sequenza lunga $n$, la complessità è $\Theta (\log^2 n)$.
    \end{solution}
\end{questions}