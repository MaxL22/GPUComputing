%!TeX spellcheck = it_IT
\section{Codice}

\begin{questions}
    \question Come usare la shared memory per le convoluzioni.
    
    \begin{solution}
        Un possibile schema per la convoluzione usando la shared memory è quello del tiling: a ogni blocco della grid viene affidato un "tile" (sotto-blocco) dell'immagine in input, ogni thread nel blocco si occupa di un prodotto della convoluzione. 
        
        In shared memory vanno caricati tutti i possibili dati a cui il blocco deve fare accesso, ovvero la porzione dell'immagine di cui si occupa, assieme all'area data dal raggio della maschera di convoluzione (un "alone" attorno al sotto-blocco stesso).
        
        Rimane il problema di come caricare i blocchi in shared memory: i dati sono più del blocco stesso, sarebbe ottimale dividere il più equamente possibile il carico di lavoro. 
        
        Un'idea può essere dare un ordine ai thread e "ripetere" il thread block sopra la sotto-matrice formata da tutti i dati da caricare. Tiling della zona da caricare in smem.
        
        L'uso della shared memory permette di limitare gli accessi alla memoria globale, incrementando la velocità degli accessi successivi, ma sono da tenere in conto possibili bank conflict (la shared memory è divisa in bank, se due o più thread vogliono accedere alla stessa bank vanno serviti serialmente) e la quantità di memoria shared utilizzata influisce sul numero di blocchi attivi concorrentemente su un singolo SM.
        
        Pseudocodice:
        \begin{center}
            \begin{minipage}{.9\textwidth}
                \begin{tcolorbox}[
                    colback=white,
                    sharp corners,
                    boxrule=.3mm,
                    left=20pt,
                    top=0pt,
                    bottom=0pt,
                    colbacktitle=white,
                    coltitle=black
                    ]
                    \LinesNumbered
                    \begin{algorithm}[H]
                        \SetAlgoNoEnd
                        \SetKwSty{texttt}
                        \SetArgSty{relax}
                        \texttt{nblocks =} numero di blocchi per il tiling;\\
                        \For{$i, j \in $ \texttt{numblocks}}{
                            \texttt{// Calcola indici} \\
                            \texttt{// Carica nella smem} \\
                        }
                        \texttt{\_\_syncthreads()}; \\
                        \texttt{sum = 0}; \\
                        \For{$i,j \in $ \texttt{MASK\_SIZE}}{
                            \texttt{sum} $\leftarrow$ \texttt{sum + smemValue * maskValue};\\
                        }
                        \texttt{// Scrivere l'output}
                    \end{algorithm}
                \end{tcolorbox}
            \end{minipage}
        \end{center}
        
        Codice:
        \begin{minted}{cuda}
__global__ void conv2D(Matrix A, Matrix B, Matrix M) {
    // Indici per colonna e riga della matrice A
    int x = blockIdx.x * blockDim.x + threadIdx.x;
    int y = blockIdx.y * blockDim.y + threadIdx.y;
    
    // Allocazione della smem
    __shared__ float smem[TILE_SIZE][TILE_SIZE];
    
    // Caricare i dati nella smem
    for (int row = 0; row <= TILE_SIZE / blockDim.y; row++) {
        for (int col = 0; col <= TILE_SIZE / blockDim.x; col++) {
            // Indici per i dati originali
            int d_row = y + blockDim.y * row - RADIUS;
            int d_col = x + blockDim.x * col - RADIUS;
            // Indici per la smem
            int s_row = threadIdx.y + blockDim.y * row;
            int s_col = threadIdx.x + blockDim.x * col;
            
            // Controlla che il range sia valido, per smem e dati
            if (s_row < TILE_SIZE && s_col < TILE_SIZE) {
                if (d_row >= 0 && d_row < A.height 
                            && d_col >= 0 && d_col < A.width) {
                    smem[s_row][s_col] =
                            A.elements[d_row * A.width + d_col];
                } else {
                    smem[s_row][s_col] = 0.0f;
                }
            }
        }
    }
    __syncthreads();
    
    // Convoluzione
    float sum = 0.0f;
    for (int i = 0; i < MASK_SIZE; i++) {
        for (int j = 0; j < MASK_SIZE; j++) {
            int r = threadIdx.y + i;
            int c = threadIdx.x + j;
            if (r >= 0 && r < TILE_SIZE && c >= 0 && c < TILE_SIZE) {
                sum += smem[r][c] * M.elements[i * MASK_SIZE + j];
            }
        }
    }
    
    // Scrive l'output
    if (y < A.height && x < A.width) {
        B.elements[y * B.width + x] = sum;
    }
}
        \end{minted}
    \end{solution}
    
    \question Pseudocodice del trasferimento in memoria shared per il prodotto matriciale.
    
    \begin{solution}
        L'idea dietro l'uso della smem è: vogliamo suddividere in tile le zone di memoria della matrici di cui fare il prodotto (divise in \texttt{nblocks} tile della dimensione del block), in modo da poterle caricare con una suddivisione del lavoro più equa possibile. Per il numero di tile \texttt{nblocks} quindi si ripete il processo di 
        \begin{itemize}
            \item caricare i valori delle due matrici in smem
            
            \item calcolare la somma parziale data dai valori parziali caricati, per ogni cella della matrice prodotto
        \end{itemize}
        
        Una volta terminato, scrivere i valori in memoria globale.
        
        Pseudocodice:
        \begin{center}
            \begin{minipage}{.9\textwidth}
                \begin{tcolorbox}[
                    colback=white,
                    sharp corners,
                    boxrule=.3mm,
                    left=20pt,
                    top=0pt,
                    bottom=0pt,
                    colbacktitle=white,
                    coltitle=black
                    ]
                    \LinesNumbered
                    \begin{algorithm}[H]
                        \SetAlgoNoEnd
                        \SetKwSty{texttt}
                        \SetArgSty{relax}
                        \texttt{\_\_shared\_\_ As[WIDTH][WIDTH]}; \\
                        \texttt{\_\_shared\_\_ Bs[WIDTH][WIDTH]}; \\
                        \texttt{nblocks =} numero di block contenuti nelle sotto-matrici da caricare; \\
                        \For{$i \in$ \texttt{nblocks}}{
                            \texttt{As[tidy][tidx] = A[tidx\_abs + $i$ * WIDTH][tidy\_abs]}; \\
                            \texttt{Bs[tidy][tidx] = B[tidx\_abs][tidy\_abs + $i$ * WIDTH]}; \\
                            \texttt{\_\_syncthreads()}; \\
                            
                            \For{$j \in$ \texttt{WIDTH}}{
                                \texttt{partial\_sum += As[tidy][j] * Bs[j][tidx]}; \\
                            }
                            \texttt{\_\_syncthreads()};\\
                        }
                        \texttt{// Scrivere il risultato in memoria globale}
                    \end{algorithm}
                \end{tcolorbox}
            \end{minipage}
        \end{center}
        
        Dove: 
        \begin{itemize}
            \item \texttt{WIDTH} è la dimensione del blocco di dati da caricare
            
            \item \texttt{tidx} e \texttt{tidy} sono le coordinate del thread all'interno del blocco
            
            \item \texttt{tidx\_abs} e \texttt{tidy\_abs} sono le coordinate del thread rispetto alla grid
            
            \item \texttt{nblocks} rappresenta il ceil della larghezza (o altezza) della matrice fratto la larghezza (o altezza) del block
        \end{itemize}
    \end{solution}
    
    \question Codice per un kernel che svolge il prodotto di matrici triangolari superiori.
    
    \begin{solution}
        Senza ottimizzazioni particolari, un kernel potrebbe essere:
        \begin{minted}{cuda}
// A e B matrici di input, C output, N dimensione
__global__ void upperTriangularMatMul(float *A, float *B, 
                float *C, int N) {
    // Indici assoluti della cella
    int row = threadIdx.y + blockIdx.y * blockDim.y;
    int col = threadIdx.x + blockIdx.x * blockDim.x;
    // Bound check
    if (row < N && col < N) {
        float sum = 0.0f;
        if (row <= col) {
            // Somma solo per k = row..col, dato che upper-triangular
            for (int k = row; k <= col; ++k) {
                sum += A[row * N + k] * B[k * N + col];
            }
        }
        // Scrivere in memoria il risultato
        C[row * N + col] = sum;
    }
}
        \end{minted} 
    \end{solution}
    
    \question Mostrare un uso pratico degli stream CUDA.
    
    \begin{solution}
        Gli stream vengono utilizzati per sovrapporre trasferimenti di memoria con computazioni da parte della GPU. Uno schema di utilizzo può essere: 
        \begin{minted}{cuda}
// Si suppongono già fatte le allocazioni di memoria

//Creazione degli stream
cudaStream_t stream1, stream2;
cudaStreamCreate(&stream1);
cudaStreamCreate(&stream2);

// Trasferimenti e lancio kernel separati
cudaMemcpyAsync(dst, src, len, cudaMemcpyHostToDevice, stream1);
cudaMemcpyAsync(dst, src, len, cudaMemcpyHostToDevice, stream2);
kernel<<<grid, block, smem, stream1>>>();
kernel<<<grid, block, smem, stream2>>>();
cudaMemcpyAsync(dst, src, len, cudaMemcpyDeviceToHost, stream1);
cudaMemcpyAsync(dst, src, len, cudaMemcpyDeviceToHost, stream1);

// Attendere che gli stream termina
cudaStreamSynchronize(stream1);
cudaStreamSynchronize(stream2);

// Rilascio delle risorse
cudaStreamDestroy(stream1);
cudaStreamDestroy(stream2);
        \end{minted}
    \end{solution}
    
    \question Uso degli stream per il prodotto di matrici diagonali a blocchi MQDB.
    
    \begin{solution}
        Gli stream sono una sequenza di operazioni CUDA incapsulate in una coda FIFO, eseguite dalla GPU in maniera asincrona rispetto all'host, sempre rispettando l'ordine di esecuzione fornito. Una MQDB è una matrice che ha valori non nulli solo in blocchi di dimensioni $d_1, \dots, d_n$ attorno alla diagonale.
        
        L'uso degli stream permette di dividere il carico di lavoro e i trasferimenti di memoria, permettendo un parallelismo tra trasferimenti ed esecuzione dei kernel (DMA permettendo). Sapendo che il prodotto di due matrici MQDB è anch'esso una matrice con le stesse proprietà, lo schema ovvio per l'utilizzo degli stream è quello di utilizzare stream diversi per ogni blocco di dati, permettendo il parallelismo tra trasferimenti e kernel.
        
        Codice (forse funzionante):
        \begin{minted}{cuda}
__global__ void streamMQDB(int *A, int *B, int *R) {
    int tid = threadIdx.x + threadIdx.y * blockDim.x;
    __shared__ int As[BS][BS];
    __shared__ int Bs[BS][BS];

    As[threadIdx.y][threadIdx.x] = A[tid];
    Bs[threadIdx.y][threadIdx.x] = B[tid];
    __syncthreads();

    float sum = 0.0f;
    for (int k = 0; k < BS; ++k) {
        sum += As[k][threadIdx.y] * Bs[threadIdx.x][k];
    }
    R[tid] = sum;
}
        \end{minted}
    \end{solution}
    
    \question Esempio e vantaggi di loop unrolling. Esempio nella reduction.
    
    \begin{solution}
        Il loop unrolling è una tecnica usata per ottimizzare i cicli: questi vengono "srotolati" in modo da ridurre l'overhead dovuto alle operazioni di controllo, ridurre il branching e i salti condizionali, aumentando così il livello di parallelismo. Si copia il corpo del loop un numero $n$ di volte, chiamato unroll factor. Lo si può fare manualmente, in alternativa, CUDA fornisce direttive di compilazione.
        
        Esempio semplice di loop unrolling: 
        \begin{minted}{cuda}
for (int i = 0; i < 4; i++){
    A[base] += B[base + i];
}
        \end{minted}
        Può diventare:
        \begin{minted}{cuda}
A[base] += B[base + 0];
A[base] += B[base + 1];
A[base] += B[base + 2];
A[base] += B[base + 3];
        \end{minted}
        
        In alternativa, usando le direttive:
        \begin{minted}{cuda}
#pragma unroll 4
for (int i = 0; i < 4; i++){
    A[base] += B[base + i];
}
        \end{minted}
        
        Lo si dovrebbe usare quando il corpo del ciclo è semplice e il numero di iterazioni è multiplo dell'unroll factor. Un unrolling troppo spinto potrebbe causare bloating del codice e riduzione delle performance dovuta a mancanza di spazio in cache/instruction buffer.
        
        Esempio nella reduction: loop all'interno del kernel, prima dell'unroll
        \begin{minted}{cuda}
for (int stride = blockDim.x / 2; stride > 0; stride >>= 1) {
    if (threadIdx.x < stride) {
        smem[threadIdx.x] += smem[threadIdx.x + stride];
    }
    __syncthreads();
}
        \end{minted}
        
        Mentre la versione con unroll è
        \begin{minted}{cuda}
if (blockDim.x >= 1024 && threadIdx.x < 512) {
    smem[threadIdx.x] += smem[threadIdx.x + 512];
    __syncthreads();
}
if (blockDim.x >= 512 && threadIdx.x < 256) {
    smem[threadIdx.x] += smem[threadIdx.x + 256];
    __syncthreads();
}
if (blockDim.x >= 256 && threadIdx.x < 128) {
    smem[threadIdx.x] += smem[threadIdx.x + 128];
    __syncthreads();
}
if (blockDim.x >= 128 && threadIdx.x < 64) {
    smem[threadIdx.x] += smem[threadIdx.x + 64];
    __syncthreads();
}
// Within a single warp
if (threadIdx.x < 32) {
    volatile float *vsmem = smem;
    vsmem[threadIdx.x] += vsmem[threadIdx.x + 32];
    vsmem[threadIdx.x] += vsmem[threadIdx.x + 16];
    vsmem[threadIdx.x] += vsmem[threadIdx.x + 8];
    vsmem[threadIdx.x] += vsmem[threadIdx.x + 4];
    vsmem[threadIdx.x] += vsmem[threadIdx.x + 2];
    vsmem[threadIdx.x] += vsmem[threadIdx.x + 1];
}
        \end{minted}
    \end{solution}
    
    \question Esempio di utilizzo della constant memory.
    
    \begin{solution}
        La constant memory è spesso utilizzata per coefficienti che devono essere letti da tutti i thread contemporaneamente. Esempio:
        \begin{minted}{cuda}
__constant__ float pi = 3.12;

// Array di cerchi, dato il raggio calcolare l'area
__global__ void circlesArea (float *area, float *radius, int N) {
    int tid = threadIdx.x + blockDim.x * blockIdx.x;
    // Bound check
    if (tid < N)
        area[tid] = radius[tid] * pi * pi;
}
        \end{minted}
    \end{solution}
\end{questions}