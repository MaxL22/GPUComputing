% !TeX spellcheck = it_IT
\section{Librerie}

\begin{questions}
    \question Schema d'uso per la libreria cuBLAS.
    
    \begin{solution}
        La libreria cuBLAS è la versione accelerata su GPU delle routine classiche di algebra lineare, come prodotti di matrici, vettori e scalari.
        
        Le funzioni hanno 3 livelli, rispettivamente per operazioni vettore-vettore, matrice-vettore e matrice-matrice.
        
        Da notare che lavora in ordine column-major (come Fortran) e non row-major (come C/C++).
        
        Uno schema d'uso tipico per usare cuBLAS è:
        \begin{enumerate}
            \item Creare un handle con \texttt{cublasCreateHandle()}
            
            \item Allocare la memoria sul dispositivo
            
            \item Popolare la device memory, ad esempio con \texttt{cublasSetVector()}
            
            \item Effettuare le chiamate a libreria per le operazioni necessarie
            
            \item Recuperare i dati dalla device memory, ad esempio con \texttt{cublasGetVector()}
            
            \item Una volta terminato, rilasciare le risorse CUDA e cuBLAS con \texttt{cudaFree()} e \texttt{cublasDestroy()}
        \end{enumerate}
    \end{solution}
    
    \question Schema d'uso per la libreria cuRAND.
    
    \begin{solution}
        La libreria cuRAND fornisce semplici generatori di numeri. Permette sequenze pseudo-random e quasi-random. Si compone di due header, la seconda è per generatori su device (opzionale).
        
        Uno schema d'uso generico di cuRAND:
        \begin{enumerate}
            \item Creare un nuovo generatore del tipo desiderato con \texttt{curandCreateGenerator()}
            
            \item Settare i parametri del generatore 
            
            \item Allocare la memoria device 
            
            \item Generare i valori casuali, ad esempio con \texttt{curandGenerate()}
            
            \item Uso dei valori 
            
            \item Quando non serve più il generatore, va distrutto con \texttt{curandDestroyGenerator()}
        \end{enumerate}
    \end{solution}
    
    \question Pseudocodice utilizzo di curand per simulare il lancio di $n$ dadi.
    
    \begin{solution}
        Seguendo l'iter tipico di utilizzo di cuRAND:
        \begin{minted}{cuda}
// Set up di cuRAND
curandCreateGenerator(&generator, CURAND_RNG_PSEUDO_DEFAULT);
curandSetPseudoRandomGeneratorSeed(generator, seed_value);

// Allocazione della memoria
cudaMalloc(&d_raw,    N * sizeof(unsigned int));
cudaMalloc(&d_dice,   N * sizeof(unsigned char));

// Generazione dei valori
curandGenerate(generator, d_raw, N);

// Kernel per normalizzare i valori nel range 1-6
kernel_normalize_values<<<blocks, threads>>>(d_raw, d_dice, N);

// Trasferimento dei valori all'host
cudaMallocHost(&h_dice, N * sizeof(unsigned char));
cudaMemcpy(h_dice, d_dice, N * sizeof(unsigned char),
cudaMemcpyDeviceToHost);

// Clean up
curandDestroyGenerator(generator);
cudaFree(d_raw);
cudaFree(d_dice);
cudaFreeHost(h_dice);
        \end{minted}
    \end{solution}
\end{questions}